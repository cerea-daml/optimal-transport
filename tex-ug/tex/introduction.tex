
\section{Introduction}
\noindent

    The transportation theory is the study of optimal transportation and allocation.
    The so-called optimal transport problem was first introduced by \cite{monge-1781} and formalized
    by \cite{kantorovich-1942}, leading to the Monge-Kantorovich transportation problem.\\

    The goal is to look for a transport map transforming a probability density function into an other
    while minimizing the cost of transport.

    \subsection{Brief description of the problem}
    \noindent

        Let \spaceE{} be an euclidean space of dimension \dimension{} and let \fI{} and \fII{} be two probability density functions over \spaceE{}.
        We are looking for a couple density-velocity \coupleFV{} defined over \EtimesT{} satisfying a continuity constraint:
        \begin{equation}
            \label{eq:continuity}
            \forall \coupleXT \in \EtimesT, \quad \frac{\partial \f}{\partial \ficttime} \coupleXT + \divergence{\f\cdot\vel} \coupleXT = 0,
        \end{equation}
        a boundary condition:
        \begin{equation}
            \label{eq:boundaries}
            \forall \x \in \spaceE, \quad \f\left(\x, \ficttime = 0\right) = \fI \andEq \f\left(\x, \ficttime = 1\right) = \fII,
        \end{equation}
        a null-flux condition for mass conservation:
        \begin{equation}
            \label{eq:null-flux}
            \forall \coupleXT \in \partial\EtimesT, \quad \vel\coupleXT = 0,
        \end{equation}
        and minimizing the cost of transport:
        \begin{equation}
            \label{eq:cost}
            \J \coupleFV = \int_{\EtimesT} \f \coupleXT \cdot \norm{ \vel \coupleXT }^2 \dx \dt.
        \end{equation}

        The minimum of this cost function is defined as the Wasserstein distance between \fI{} and \fII{}. The argument \fstar{} of the
        minimum defines a non-trivial interpolation between \fI{} and \fII{}.
        For more details on this problem, see \cite{benamou-2000, villani-2008, farchi-2016}.

    \subsection{About the solver}
    \noindent

        The optimal transport problem is a minimization problem under constraint. Following \cite{papadakis-2014}, we propose iterative
        solvers that rely on the use of proximal operators. To be able to numerically represent the fields, we have to assume that
        \spaceE{} is compact, and in particular we take $\spaceE=[0,1]^{\dimension}$. Moreover, to simplify the computation, we will work with
        the variables \coupleFM{}, where $\mom=\f\vel$ represents the momentum.\\

        In the \python module we implemented the Douglas-Rachford (\drAlgo{}) and the Primal-Dual (\pdAlgo{}) algorithms in order to compute the density
        and momentum fields satisfying \eqref{eq:continuity} and \eqref{eq:boundaries} and minimizing \eqref{eq:cost}
        given two tables representing discretized versions of
        the probability density functions \fI{} and \fII{}. Our \python code can handle the one-dimensional and two-dimensional cases
        (\ie{} $\dimension=1$ or $2$), is object-oriented, heavily relies on the standard \numpy{} and \scipy{} modules 
        (\eg{} linalg, fftpack, tensordot...) and is not parallelized. It is also able to solve the optimal transport
        problem with reservoir boundaries as presented in \cite{farchi-2016}.\\

        The code has been originally written for \python{2.7}. An other version is available for \python{3.4} which only
        introduces minor modifications to match the new \python{3.x} syntax, however this version is slightly slower. This is due to the
        lower performance of the \numpy{} and \scipy{} modules in \python{3.4}.
        For more details about the code see \cite{farchi-2016}.

