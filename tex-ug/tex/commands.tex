% Personal commands

\newcommand{\er}{\ensuremath{^{\text{er}}}}
\newcommand{\eme}{\ensuremath{^{\text{\`eme}}}}
\newcommand{\ie}{\textit{i.e.}}
\newcommand{\eg}{\textit{e.g.}}
\newcommand{\NB}{\textit{N.B.}}
\newcommand{\apriori}{\textit{a priori}}
\newcommand{\aposteriori}{\textit{a posteriori}}
\newcommand{\us}{\textunderscore}
\newcommand{\inG}[1]{"#1"}

\newcommand{\HRule}{\rule{\linewidth}{0.5mm}}
\newcommand{\hRule}{\rule{\linewidth}{0.2mm}}

\newcommand{\precite}[1]{\textcolor{red}{#1}}
\newcommand{\notes}[1]{\textcolor{green}{#1}}
\newcommand{\pemph}[1]{\textit{\textbf{#1}}}

% Numerics

\newcommand{\E}[1]{\ensuremath{\cdot 10^{#1}}}
\newcommand{\numVal}[1]{\textcolor{green}{#1}}
\newcommand{\numUnit}[1]{\textcolor{green}{#1}}

% Colors

\definecolor{codegreen}{rgb}{0,0.6,0}
\definecolor{codegray}{rgb}{0.5,0.5,0.5}
\definecolor{codepurple}{rgb}{0.58,0,0.82}
\definecolor{codebkg}{rgb}{0.95,0.95,0.92}

% Encoding 

\DeclareUnicodeCharacter{00A0}{~}

% For path

\newcommand{\ppath}[1]{{\ttfamily #1}}

% For code

\newcommand{\pcode}[1]{{\ttfamily #1}}
\newcommand{\bcode}[1]{{\ttfamily #1}}

% Specific for this document

\newcommand{\python}[1]{\emph{python} #1}
\newcommand{\numpy}{\emph{numpy}}
\newcommand{\scipy}{\emph{scipy}}
\newcommand{\cpck}{\emph{cPickle}}
\newcommand{\mpl}{\emph{matplotlib}}
\newcommand{\ffmpeg}{\emph{ffmpeg}}
\newcommand{\mencoder}{\emph{mencoder}}

\newcommand{\pdAlgo}{PD}
\newcommand{\drAlgo}{DR}

\newcommand{\spaceE}{\ensuremath{\mathbb{E}}}
\newcommand{\EtimesT}{\ensuremath{\mathbb{E}\times\left[0,1\right]}}
\newcommand{\dimension}{\ensuremath{d}}
\newcommand{\fI}{\ensuremath{f_0}}
\newcommand{\fII}{\ensuremath{f_1}}
\newcommand{\f}{\ensuremath{f}}
\newcommand{\fstar}{\ensuremath{f^*}}
\newcommand{\vel}{\ensuremath{\mathbf{v}}}
\newcommand{\mom}{\ensuremath{\mathbf{m}}}
\newcommand{\coupleFV}{\ensuremath{\left(\f,\vel\right)}}
\newcommand{\coupleFM}{\ensuremath{\left(\f,\mom\right)}}
\newcommand{\x}{\ensuremath{\mathbf{x}}}
\newcommand{\ficttime}{\ensuremath{t}}
\newcommand{\coupleXT}{\ensuremath{\left(\x,\ficttime\right)}}
\newcommand{\divergence}[1]{\ensuremath{\mathrm{div}\left(#1\right)}}
\newcommand{\andEq}{\quad\text{and}\quad}
\newcommand{\J}{\ensuremath{J}}
\newcommand{\norm}[1]{\ensuremath{\left\|#1\right\|}}
\newcommand{\dx}{\ensuremath{{\rm d}\x}}
\newcommand{\dt}{\ensuremath{{\rm d}\ficttime}}

\newcommand{\proximal}[2]{\ensuremath{\mathrm{Prox}_{#1}\left(#2\right)}}

% For algorithms

\newcommand{\costToMin}{\ensuremath{F}}
\newcommand{\costIToMin}{\ensuremath{F_1}}
\newcommand{\costIIToMin}{\ensuremath{F_2}}
\newcommand{\costIIIToMin}{\ensuremath{F_3}}
\newcommand{\linearOperator}{\ensuremath{A}}

\newcommand{\pdStateU}[1]{\ensuremath{u_{#1}}}
\newcommand{\pdStateY}[1]{\ensuremath{y_{#1}}}
\newcommand{\pdStateV}[1]{\ensuremath{v_{#1}}}

\newcommand{\pdParSigma}{\ensuremath{\sigma}}
\newcommand{\pdParTau}{\ensuremath{\tau}}
\newcommand{\pdParTheta}{\ensuremath{\theta}}

\newcommand{\drTStateUI}[1]{\ensuremath{u^1_{#1}}}
\newcommand{\drTStateUII}[1]{\ensuremath{u^2_{#1}}}
\newcommand{\drTStateUIII}[1]{\ensuremath{u^3_{#1}}}
\newcommand{\drTStateX}[1]{\ensuremath{x_{#1}}}
\newcommand{\drTStateP}{\ensuremath{p}}
\newcommand{\drTStatePI}{\ensuremath{p^1}}
\newcommand{\drTStatePII}{\ensuremath{p^2}}
\newcommand{\drTStatePIII}{\ensuremath{p^3}}

\newcommand{\drTParGamma}{\ensuremath{\gamma}}
\newcommand{\drTParAlpha}{\ensuremath{\alpha}}
\newcommand{\drTParOmegaI}{\ensuremath{\omega_1}}
\newcommand{\drTParOmegaII}{\ensuremath{\omega_2}}
\newcommand{\drTParOmegaIII}{\ensuremath{\omega_3}}

% Listings
\newcommand{\codeNumericsStyle}{\color{red}}
\lstdefinestyle{codepython}{
    backgroundcolor   = \color{codebkg},         % choose the background color; you must add \usepackage{color} or \usepackage{xcolor}
    basicstyle        = \footnotesize\ttfamily,  % the size of the fonts that are used for the code
    breakatwhitespace = false,                   % sets if automatic breaks should only happen at whitespace
    breaklines        = true,                    % sets automatic line breaking
    captionpos        = b,                       % sets the caption-position to bottom
    commentstyle      = \color{codegreen},       % comment style
    deletekeywords    = {...},                   % if you want to delete keywords from the given language
    escapeinside      = {\%*}{*)},               % if you want to add LaTeX within your code
    extendedchars     = true,                    % lets you use non-ASCII characters; for 8-bits encodings only, does not work with UTF-8
    frame             = single,                  % adds a frame around the code
    keepspaces        = true,                    % keeps spaces in text, useful for keeping indentation of code (possibly needs columns=flexible)
    keywordstyle      = \bfseries\color{blue},     % keyword style
    language          = Python,                  % the language of the code
    otherkeywords     = {numpy, np, 
        print, self},                            % if you want to add more keywords to the set
    numbers           = left,                    % where to put the line-numbers; possible values are (none, left, right) 
    numbersep         = 5pt,                     % how far the line-numbers are from the code
    numberstyle       = \tiny{\color{codegray}}, % the style that is used for the line-numbers
    rulecolor         = \color{black},           % if not set, the frame-color may be changed on line-breaks within not-black text (e.g. comments (green here))
    showspaces        = false,                   % show spaces everywhere adding particular underscores; it overrides 'showstringspaces'
    showstringspaces  = false,                   % underline spaces within strings only
    showtabs          = false,                   % show tabs within strings adding particular underscores
    stepnumber        = 5,                       % the step between two line-numbers. If it's 1, each line will be numbered
    stringstyle       = \color{codepurple},      % string literal style
    tabsize           = 4,                       % sets default tabsize to 2 spaces
    %caption           = \lstname,                % show the filename of files included with \lstinputlisting; also try caption instead of title
    %identifierstyle   = \color{blue},
    literate          =
        *{0}{{{\codeNumericsStyle0}}}{1}
        {1}{{{\codeNumericsStyle1}}}{1}
        {2}{{{\codeNumericsStyle2}}}{1}
        {3}{{{\codeNumericsStyle3}}}{1}
        {4}{{{\codeNumericsStyle4}}}{1}
        {5}{{{\codeNumericsStyle5}}}{1}
        {6}{{{\codeNumericsStyle6}}}{1}
        {7}{{{\codeNumericsStyle7}}}{1}
        {8}{{{\codeNumericsStyle8}}}{1}
        {9}{{{\codeNumericsStyle9}}}{1}
        {.0}{{{\codeNumericsStyle.0}}}{2}
        {.1}{{{\codeNumericsStyle.1}}}{2}
        {.2}{{{\codeNumericsStyle.2}}}{2}
        {.3}{{{\codeNumericsStyle.3}}}{2}
        {.4}{{{\codeNumericsStyle.4}}}{2}
        {.5}{{{\codeNumericsStyle.5}}}{2}
        {.6}{{{\codeNumericsStyle.6}}}{2}
        {.7}{{{\codeNumericsStyle.7}}}{2}
        {.8}{{{\codeNumericsStyle.8}}}{2}
        {.9}{{{\codeNumericsStyle.9}}}{2}
        {\ }{{ }}{1}
}

\lstdefinestyle{codebash}{
    backgroundcolor   = \color{codebkg},         % choose the background color; you must add \usepackage{color} or \usepackage{xcolor}
    basicstyle        = \footnotesize\ttfamily,  % the size of the fonts that are used for the code
    breakatwhitespace = false,                   % sets if automatic breaks should only happen at whitespace
    breaklines        = true,                    % sets automatic line breaking
    captionpos        = b,                       % sets the caption-position to bottom
    commentstyle      = \color{codegreen},       % comment style
    deletekeywords    = {...},                   % if you want to delete keywords from the given language
    escapeinside      = {\%*}{*)},               % if you want to add LaTeX within your code
    extendedchars     = true,                    % lets you use non-ASCII characters; for 8-bits encodings only, does not work with UTF-8
    frame             = single,                  % adds a frame around the code
    keepspaces        = true,                    % keeps spaces in text, useful for keeping indentation of code (possibly needs columns=flexible)
    keywordstyle      = \textbf{\color{blue}},   % keyword style
    language          = sh,                      % the language of the code
    otherkeywords     = {},                      % if you want to add more keywords to the set
    numbers           = none,                    % where to put the line-numbers; possible values are (none, left, right) 
    numbersep         = 5pt,                     % how far the line-numbers are from the code
    numberstyle       = \tiny{\color{codegray}}, % the style that is used for the line-numbers
    rulecolor         = \color{black},           % if not set, the frame-color may be changed on line-breaks within not-black text (e.g. comments (green here))
    showspaces        = false,                   % show spaces everywhere adding particular underscores; it overrides 'showstringspaces'
    showstringspaces  = false,                   % underline spaces within strings only
    showtabs          = false,                   % show tabs within strings adding particular underscores
    stepnumber        = 2,                       % the step between two line-numbers. If it's 1, each line will be numbered
    stringstyle       = \color{codepurple},      % string literal style
    tabsize           = 2,                       % sets default tabsize to 2 spaces
    %caption           = \lstname,                % show the filename of files included with \lstinputlisting; also try caption instead of title
    %identifierstyle   = \color{blue},
    literate          =
        *{0}{{{\codeNumericsStyle0}}}{1}
        {1}{{{\codeNumericsStyle1}}}{1}
        {2}{{{\codeNumericsStyle2}}}{1}
        {3}{{{\codeNumericsStyle3}}}{1}
        {4}{{{\codeNumericsStyle4}}}{1}
        {5}{{{\codeNumericsStyle5}}}{1}
        {6}{{{\codeNumericsStyle6}}}{1}
        {7}{{{\codeNumericsStyle7}}}{1}
        {8}{{{\codeNumericsStyle8}}}{1}
        {9}{{{\codeNumericsStyle9}}}{1}
        {.0}{{{\codeNumericsStyle.0}}}{2}
        {.1}{{{\codeNumericsStyle.1}}}{2}
        {.2}{{{\codeNumericsStyle.2}}}{2}
        {.3}{{{\codeNumericsStyle.3}}}{2}
        {.4}{{{\codeNumericsStyle.4}}}{2}
        {.5}{{{\codeNumericsStyle.5}}}{2}
        {.6}{{{\codeNumericsStyle.6}}}{2}
        {.7}{{{\codeNumericsStyle.7}}}{2}
        {.8}{{{\codeNumericsStyle.8}}}{2}
        {.9}{{{\codeNumericsStyle.9}}}{2}
        {\ }{{ }}{1}
}

% Environments

\newenvironment{tablehtbp}{
    \begin{table}[htbp]
        \centering
        \scriptsize}{
    \end{table}}
